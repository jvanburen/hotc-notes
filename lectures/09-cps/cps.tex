\documentclass{amsart}
\usepackage{amsmath, amssymb, latexsym}
\usepackage{bussproofs}
\usepackage{mathpartir, proof}
\usepackage{stackengine}
\usepackage{xcolor}
\makeatletter
\renewcommand\subsection{\@startsection{subsection}{2}%
  \z@{-.5\linespacing\@plus-.7\linespacing}{.5\linespacing}%
  {\normalfont\scshape}}
\renewcommand\subsubsection{\@startsection{subsubsection}{3}%
  \z@{.5\linespacing\@plus.7\linespacing}{-.5em}%
  {\normalfont\scshape}}
\makeatother
\title{Continuation Passing Style}

\newcommand{\type}{\ensuremath{\mathtt{type}}}
\newcommand{\kind}{\ensuremath{\mathtt{kind}}}
\newcommand{\exn}{\ensuremath{\mathtt{exn}}}
\renewcommand{\raise}{\ensuremath{\mathtt{raise}}}
\newcommand{\handle}{\ensuremath{\mathtt{handle}}}
\newcommand{\pack}{\text{\texttt{pack} }}
\newcommand{\unpack}{\text{\texttt{unpack} }}
\newcommand{\iin}{\text{ \texttt{in} }}
\newcommand{\halt}{\text{\texttt{halt}}}
\newcommand{\llet}{\text{\texttt{let} }}
\newcommand{\as}{\text{ \texttt{as} }}
\newcommand{\src}[1]{{\color{purple}#1}}

\begin{document}
\maketitle

After CPS conversion, we will resolutely use continuations for everything. This can be seen as a way of making control flow explicit. There are results saying that the output of CPS conversion is invariant under interpretation as pass-by-name or pass-by-value, though we will not go into those results in this class. CPS conversion gives us named intermediate results. Thirdly, we reify control-flow as data. The first two of these three properties are commonly called ``monadic form.''

\section{IL-CPS}

We first must define the target language for this transformation. Notably, we split terms into two syntatic classes; \emph{expressions} and \emph{values}. One may think of expressions as values that are computed and then thrown away.

We may formalize this intuition as follows:
\begin{align*}
v ::&= x\\
&\mid \lambda x : \tau . e\\
&\mid \pack [c,v] \as \exists \alpha : k .\tau\\
&\mid \langle v_1, \ldots v_n \rangle\\[1ex]
e ::&= v v \\
&\mid \unpack [\alpha, x] = v \iin e\\
&\mid \llet x = \pi_i v \iin e\\
&\mid \llet x = v \iin e\\
&\mid \halt
\end{align*}

IL-CPS has the following typing rules:
\begin{mathpar}
\inferrule{\Gamma \vdash \tau : T \\
           \Gamma, x:\tau \vdash e : 0}{%
           \Gamma \vdash \lambda x : \tau . e : \tau \rightarrow 0}\and
\inferrule{\Gamma \vdash v_1 : \neg \tau \\
           \Gamma \vdash v_2 : \tau}{%
           \Gamma \vdash v_1 v_2 : 0}\and
\inferrule{\Gamma \vdash c : k \\
           \Gamma \vdash v : [c/\alpha]\tau\\
           \Gamma, \alpha:k\vdash \tau:T}{%
           \Gamma \vdash \pack [c,v] \as \exists\alpha : k . \tau : \exists \alpha : k . \tau}\and
\inferrule{\Gamma \vdash v : \exists \alpha : k . \tau \\
           \Gamma, \alpha:k, x : \tau \vdash e : 0}{%
           \Gamma \vdash \unpack [\alpha,x] = v \iin e : 0}\and
\inferrule{\Gamma \vdash v_i : \tau_i\\\text{(for $i = 1\ldots n$)}}{%
           \Gamma \vdash \langle v_1,\ldots,v_n \rangle : \times[\tau_1,\ldots,\tau_n]}\and
\inferrule{\Gamma \vdash v : \times[\tau_1,\ldots,\tau_n]}{%
           \Gamma \vdash \llet x = \pi_i v \iin e : 0}\and
\inferrule{\Gamma \vdash v : \tau\\
           \Gamma, x: \tau \vdash e : 0}{%
           \Gamma \vdash \llet x = v \iin e : 0}\and
\inferrule{ }{\Gamma \vdash \halt : 0}\and
\inferrule{\Gamma \vdash \tau : T }{\Gamma \vdash \neg \tau : T}\and
\end{mathpar}
Note that in constructive logic, the proposition ``$\tau \rightarrow 0$'' is exactly $\neg\tau$. So we may perhaps cloyingly say that continuations are negation.

A careful reader may notice our usual sleight of hand in the \texttt{unpack} rule: the $\alpha$'s mentioned are all asserted to be equal.

\section{CPS Conversion: Compiler Pass}

Kind, constructor, and type translation are all still syntax-directed. Most every transformation is an identity mapping, with one exception:
\[
\src{\tau_1\rightarrow\tau_2} = \neg(\src{\tau_1} \times \neg \src{\tau_2}).
\]

There's a neat connection to constructive logic here; by the Curry-Howard Isomorphism, this is analogous to the transformation $A\supset B$ goes to $\neg(A\land \neg B)$. We're effectively DeMorgan-ing our code here.

Context translation is just the usual map of kind and type translation.

\subsection{Transforming Terms}

We have
\[
\Gamma\vdash e :\tau \rightarrow x.\src{e}
\]
Here, $\src{e}$ is a continuation that passes its value to the bound variable $x$. We maintain the invariant that ``If $\Gamma\vdash e:\tau \rightarrow x.\src{e}$, then $\src{\Gamma}x:\neg\src{\tau}\vdash\src{e}:0$.''

In respect of convention, we'll strive to use the variable $k$ instead of $x$ as the continuation variable here. One hopes that this does not cause the reader any great difficulty, as we also often use the variable $k$ for kinds.

\begin{mathpar}
\inferrule{%
\Gamma(x)=\tau
  }{%
\Gamma\vdash x : \tau \leadsto k.(k x)
}\and

% TFW your latex source is more readable than the output pdf
\inferrule{%
\Gamma \vdash e : \times[\tau_0,\ldots \tau_{n-1}] \leadsto k'.\src{e}
  }{%
\Gamma\vdash \pi_i(e) : \tau_i \leadsto k.
  (\llet k' = (
    \lambda x : \times [\src{\tau_0},\ldots,\src{\tau_{n-1}}]. \llet y = \pi_i k \iin k y)
  \iin \src{e})
}\and
\end{mathpar}

\begin{mathpar}
\inferrule{%
\Gamma \vdash e_i : \tau_i \leadsto k_i . \src{e_i} \\ \text{(for $i=1,\ldots,n$)}
  }{%
\Gamma\vdash \langle e_1,\ldots e_n \rangle : \times [\tau_1,\ldots\tau_n] \leadsto k.\left(
\begin{tabular}{l}
  $\llet k_1 = (\lambda x_i : \src{\tau_1}.$\\
  $\llet k_2 = (\lambda x_i : \src{\tau_2}. \ldots$\\
  $\llet k_n = (k \langle x_1, \ldots x_n \rangle) \iin \src{e_n}) \iin \src{e_{n-1}})$\\
  $\text{\texttt{in} } \ldots) \iin \src{e_2}) \iin \src{e_1})$
\end{tabular}\right)
}\and
\end{mathpar}

\begin{mathpar}
\inferrule{%
\Gamma \vdash \tau_1 : T \\ \Gamma, x:\tau_1 \vdash e : \tau_2 \leadsto k'.\src{e}
  }{%
\Gamma\vdash \lambda x : \tau_1 .e : \tau_1 \rightarrow \tau_2 \leadsto k. k\left(
\begin{tabular}{l}
  $\lambda y: \src{\tau_1}\times\neg\src{\tau_2}.$\\
  $\llet x = \pi_0 y \iin$\\
  $\llet k' = \pi_1 y \iin \src{e}$\\
\end{tabular}\right)
}\and

\inferrule{%
\Gamma \vdash e_1: \tau\rightarrow\tau' \leadsto k_1. \src{e_1} \\
\Gamma \vdash e_2: \tau\rightarrow\tau' \leadsto k_2. \src{e_2}
  }{%
\Gamma\vdash e_1 e_2 : \tau' \leadsto k.\left(
\begin{tabular}{l}
  $\llet k_1 = \textbf{(}\lambda f : \neg(\src{\tau}\times\neg\src{\tau'}).$\\
  $\llet k_2 = (\lambda x : \src{\tau}. f \langle x, k\rangle)\iin \src{e_2}\textbf{)}$\\
  $\text{\texttt{in} }\src{e_1}$
\end{tabular}\right)
}\and

\inferrule{%
\Gamma\vdash c : k \\
\Gamma\vdash e : [c/\alpha]\tau \leadsto k'.\src{e}\\
\Gamma, \alpha : k\vdash\tau : T
  }{%
\Gamma\vdash \pack [c,e] \as \exists\alpha:k.\tau : \exists\alpha:k.\tau\leadsto k.\left(
\begin{tabular}{l}
  $\llet k' = $\\
  $\texttt{  }\lambda x : [c/\alpha]\src{e}. k\left(\pack [\src{c},x] \as \exists\alpha:\src{k}.\src{\tau}\right)$\\
  $\text{\texttt{in} }\src{e}$
\end{tabular}\right)
}\and

\inferrule{%
\Gamma\vdash e_1 : \exists \alpha : k. \tau \leadsto k_1 . \src{e_1}\\
\Gamma, \alpha : k, x : \tau\vdash e_1 : e_2 : \tau' \leadsto k_2.\src{e_2}\\
  }{%
\Gamma\vdash \unpack [\alpha,x] = e_1 \iin e_2 \leadsto k.\left(
\begin{tabular}{l}
  $\llet k_1 = \lambda x_1 : (\exists\alpha:\src{k}.\src{\tau}).$\\
  $\texttt{  }\left(\unpack [\alpha, x] = x_1 \iin (\llet k_2 = \lambda x_2 : \tau'. k x_2 \iin \src{e_2})\right)$\\
  $\text{\texttt{in} }\src{e_1}$
\end{tabular}\right)
}\and

\inferrule{%
\Gamma\vdash k : \kind\\
\Gamma, \alpha : k \vdash e : \tau \leadsto k'.\src{e}
  }{%
\Gamma\vdash \Lambda \alpha:k.e : \forall \alpha : k.c \leadsto k. k \left(
\begin{tabular}{l}
  $(\lambda x : (\exists\alpha : \src{k}.\neg\src{\tau}).$\\
  $\unpack [\alpha, k'] = x \iin \src{e})$
\end{tabular}\right)
}\and

\inferrule{%
\Gamma\vdash \forall\alpha : k . \tau \leadsto k'.\src{e} \\
\Gamma\vdash c : k
  }{%
\Gamma\vdash e[c] : [c/\alpha]\tau\leadsto k.\left(
\begin{tabular}{l}
  $\llet k' = \lambda f : (\neg(\exists \alpha : \src{k} : \neg \src{\tau})).$\\
  $f (\pack [\src{c}, k] \as \exists \alpha : \src{k}.\neg\src{\tau})$\\
  $\text{\texttt{in} } \src{e}$
\end{tabular}\right)
}\and


\end{mathpar}
We also have the following type transformations:
\begin{align*}
\src{\tau_1\rightarrow\tau_2} &= \neg(\src{\tau_1}\times\neg\src{\tau_2})\\
\src{\forall\alpha : k.\tau} &= \neg(\exists\alpha:\src{k}.\neg\src{\tau})
\end{align*}
We won't talk about sums, references, exns, primitives, or recursive types here. These are left as an exercise for the reader.

\subsection{Exceptions}

We're basically just going to go through everything and redo it. However, the rewrites are pretty straightforward - we're basically just going to pass a failure continuation in through everything.

We have new type transformations:
\begin{align*}
\src{\tau_1\rightarrow\tau_2} &= \neg(\times[\src{\tau_1},\neg\src{\tau_2}, \neg\exn])\\
\src{\forall\alpha : k.\tau} &= \neg(\exists\alpha:\src{k}.(\neg\src{\tau}\times \neg \exn))
\end{align*}

We also change our judgement to have the form
\[
\Gamma\vdash e : \tau \leadsto k k_{ex} . \src{e}
\]

Most rules remain largely unchanged, just pushing the failure continuation through. For instance,
\begin{mathpar}

\inferrule{%
\Gamma(x) = \tau
  }{%
\Gamma\vdash x : \tau \leadsto k k_{ex} . k x
}\and

\inferrule{%
\Gamma \vdash e_i : \tau_i \leadsto k_i k_{ex_i}. \src{e_i} \\ \text{(for $i=1,\ldots,n$)}
  }{%
\Gamma\vdash \langle e_1,\ldots e_n \rangle : \times [\tau_1,\ldots\tau_n] \leadsto k k_{ex}.\left(
\begin{tabular}{l}
  $\llet k_1 = (\lambda x_i : \src{\tau_1}.$\\
  $\llet k_2 = (\lambda x_i : \src{\tau_2}. \ldots$\\
  $\llet k_n = (k \langle x_1, \ldots x_n \rangle) \iin \src{e_n}) \iin \src{e_{n-1}})$\\
  $\text{\texttt{in} } \ldots) \iin \src{e_2}) \iin \src{e_1})$
\end{tabular}\right)
}\and

\end{mathpar}
However, the rules for exceptional control flow have yet to be defined:
\begin{mathpar}

\inferrule{%
\Gamma\vdash\tau : T\\
\Gamma\vdash e : \exn \leadsto k' k'_{ex}.\src{e}
  }{%
\Gamma\vdash \raise_\tau e : \tau \leadsto k k_{ex}. \left(
\begin{tabular}{l}
  $\llet k' = (\lambda x : \exn . (k_{ex} x))$\\
  $\text{\texttt{in} }\llet k'_{ex} = k_{ex}$\\
  $\text{\texttt{in} } \src{e}$
\end{tabular}\right)
}\and

\inferrule{%
\Gamma\vdash e_1 : \tau \leadsto k_1 k_{ex_1} . \src{e_1}\\
\Gamma, x:\exn\vdash e_2 : \tau \leadsto k_2 k_{ex_2} . \src{e_2}
  }{%
\Gamma\vdash \handle(e_1, x.e_2) : \tau \leadsto k k_{ex}. \left(
\begin{tabular}{l}
  $\llet k_1 = k \iin$\\
  $\llet k_{ex} = (\lambda x : \exn . $\\
  $\texttt{  }\llet k_2 = k \iin \llet k_{ex_2} = k_{ex} \iin \src{e_2})$\\
  $\text{\texttt{in} } \src{e_1}$
\end{tabular}\right)
}\and

\inferrule{%
\Gamma\vdash \tau_1 : T \\
\Gamma, x : \tau_1 \vdash e : \tau_2 \leadsto k' k'_{ex}. \src{e}
  }{%
\Gamma\vdash \lambda x : \tau_1. e : (\tau_1\rightarrow\tau_2) \leadsto k k_{ex}. k \left(
\begin{tabular}{l}
  $\lambda (y : \times[\src{\tau_1}, \neg\src{\tau_2}, \neg\exn]).$\\
  $\llet x = \pi_0 y$\\
  $\llet k' = \pi_1 y$\\
  $\llet k'_{ex} = \pi_2 y$\\
  $\text{\texttt{in} } \src{e}$
\end{tabular}\right)
}\and

\inferrule{%
\Gamma\vdash e_1 : \tau\rightarrow\tau' \leadsto k_1 k_{ex}. \src{e_1}\\
\Gamma\vdash e_2 : \tau \leadsto k_2 k_{ex}. \src{e_2}
  }{%
\Gamma\vdash e_1 e_2 : \tau' \leadsto k k_{ex}. \left(
\begin{tabular}{l}
  $\llet k_1 (\lambda f : \neg(\times[\src{\tau}, \neg\src{\tau'}, \neg\exn]).$\\
  $\llet k_2 = (\lambda x : \src{\tau}. f \langle x, k, k_{ex} \rangle)$\\
  $\text{\texttt{in} } \src{e_2}) \iin \src{e_1}$\\
\end{tabular}\right)
}\and

\end{mathpar}
\end{document}
