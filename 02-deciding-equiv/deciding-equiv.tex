\documentclass{amsart}
\usepackage{amsmath, amssymb, latexsym}
\usepackage{mathpartir, proof}
\title{Deciding constructor equivalence}

\newcommand{\type}{\ensuremath{\mathtt{type}}}

\begin{document}
\maketitle

\begin{abstract}
    In the previous section we described definitional equality for constructors. The presentation given the most natural definition of the operator $\equiv$, but is not so amenable to translation to code, which will complicate the implementation of type-checking. Here we will describe a couple of algorithms to decide whether $\Gamma \vdash c_1 \equiv c_2 : k$. The first, {\bf normalize-and-compare} is conceptually appealing but doesn't scale to some richer calculi we will cover. So we will develop a second approach, called {\bf algorithmic constructor equivalence}, which is an inductively defined judgement $\Gamma \vdash c_1 \iff c_2 : k$ that better lends itself to implementation in code.
\end{abstract}

\begin{section}{Normalize-and-compare}
    \newcommand{\normal}[1]{#1 \ \mathsf{normal}}
    Recall the syntactic specification of (souped-up) $F^\omega$.
    \[
      \begin{array}{lcl}
        k & ::= & \type \mid k \to k \mid k * k\\
        c & ::= & \alpha \mid c \to c \mid \forall \alpha : k. c
            \mid \lambda \alpha : k. c \mid c\ c \mid \langle c, c \rangle \mid \pi_1 \ c \mid \pi_2 \ c \\
        e & ::= & x \mid \lambda x : c. e \mid e\ e \mid
                \Lambda \alpha : k. e \mid e[c]\\
        \Gamma & ::= & \cdot \mid \Gamma, x : c \mid \Gamma, \alpha : k
      \end{array}
    \]

    The idea of normalize-and-compare is simple. We would give a pair of inductively defined judgements, $\Gamma \vdash \normal{c}$ and $\Gamma \vdash c \leadsto c'$, which would be defined such that if $\vdash c : k$ (that is, $c$ is well-formed) then $c$ would eventually step $\leadsto$ to some $c'$ with $\normal{c'}$ - the idea being that normalized constructors are easy to check for equality. Examples of the rules defining these judgements include
        \begin{mathpar}
            \inferrule[1a]{\quad}{\Gamma, \alpha : \type \vdash \normal{\alpha}} \and
            \inferrule[1b]{\Gamma \vdash \normal{c_1} \quad \Gamma \vdash \normal{c_2}}{\Gamma \vdash \normal{\langle c_1, c_2 \rangle}} \and
            \inferrule[1c]{\Gamma, \alpha : k \vdash \normal c}{\Gamma \vdash \normal{(\forall \alpha : k. \; c)}} \and
            \inferrule[1d]{\Gamma \vdash c \leadsto c'}{\Gamma \vdash \pi_1 \ c \leadsto \pi_1 \ c'} \and
            \inferrule[1e]{\Gamma \vdash \normal{c_2} \quad \Gamma \vdash c_2 : k}{\Gamma \vdash (\lambda \alpha : k. \ c_1) \ c_2 \leadsto [c_2/\alpha]c_1}
        \end{mathpar}

    And so forth. Additionally, we specify a judgement for the transitive closure of $\leadsto$:
        \begin{mathpar}
            \inferrule[1f]{c \leadsto c' \quad c' \leadsto^* c''}{c \leadsto^* c''}
            \and
            \inferrule[1g]{\normal{c}}{c \leadsto^* c}
        \end{mathpar}
    at last we would develop a judgement $\Gamma \vdash c_1 \equiv_n c_2 : k$, which would be a simple structural equivalence comparison on normalized constructors. Then the algorithm to check $\Gamma \vdash c_1 \equiv c_2 : k$ is as follows.
        \begin{enumerate}
            \item Determine whether $\vdash c_1 : k_1$ and $\vdash c_2 : k_2$ - that is, that $c_1$ and $c_2$ are well-formed constructors of some kinds $k_1$ and $k_2$ respectively.
            \item If so, check whether $k_1$ is $k_2$; this is a simple comparison because we don't have a complicated kind structure.
            \item If so, compute $c_1'$ and $c_2'$ such that $c_1 \leadsto^* c_1'$ and $c_2 \leadsto^* c_2'$.
            \item Determine whether $\Gamma \vdash c_1' \equiv_n c_2' : k$.
        \end{enumerate}

    The reader may supply the remaining rules. We are going to choose to focus on a different algorithm however, because eventually we will cover the ``singleton-kind calculus,'' which doesn't play nice with normalize-and-compare.
    \section{Algorithmic Constructor Equivalence}
    The goal now is define algorithmic constructor equivalence, which is specified by the judgement $\Gamma \vdash c_1 \iff c_2 : k$. In order to define it we will use a number of auxiliary judgements, which are summarized in the table below. I annotated the judgements with polarities $+$ and $-$, which indicate whether the corresponding variable should be an input or an output, respectively, when the judgement is implemented in code. Implementation of judgements which have no $-$ variables simply determine whether the judgement is derivable.
        \newcommand{\patheq}{\ \longleftrightarrow\ }
        \newcommand{\whnorm}{\Downarrow}
        \newcommand{\whred}{\leadsto}
        \newcommand{\synth}{\Rightarrow}
        \renewcommand{\check}{\Leftarrow}
        \[
            \begin{tabular}{l l}
                Judgement & Description \\
                \hline
                $\Gamma^+ \vdash c_1^+ \iff c_2^+ : k^+$ & Algorithmic constructor equivalence \\
                $\Gamma^+ \vdash c_1^+ \patheq c_2^+ : k^-$ & Algorithmic path equivalence \\
                $c^+ \whnorm n^-$ & Weak-head normalization \\
                $c^+ \whred c'^-$ & Weak-head reduction
            \end{tabular}
        \]

    The idea of this algorithm is similar to normalize-and-compare, but it is based on ``weak-head normalization'' rather than plain normalization. The difference will become clear shortly. The principal rules defining the judgement $\Gamma \vdash c_1 \iff c_2 : k$ are

        \begin{mathpar}
            \inferrule[2a]{\Gamma, \alpha : k_1 \vdash c \ \alpha \iff c' \ \alpha : k_2}{\Gamma \vdash c \iff c' : k_1 \to k_2} \and
            \inferrule[2b]{\Gamma \vdash \pi_1 \ c \iff \pi_1 \ c' : k_1 \quad \Gamma \vdash \pi_2 \ c \iff \pi_2 \ c' : k_2 }{\Gamma \vdash c \iff c' : k_1 * k_2} \and
            \inferrule[2c]{c_1 \whnorm n_1 \quad c_2 \whnorm n_2 \quad \Gamma \vdash n_1 \patheq n_2 : \type}{\Gamma \vdash c_1 \iff c_2 : \type}
        \end{mathpar}

    $2c$ is the most interesting rule here. We will define a judgement $\Gamma \vdash n_1 \patheq n_2 : k$ which accepts weak-head normalized versions of given constructors and uses them to decide equivalence. The judgement $c \whnorm n$ specifies that $n$ is the weak-head normalization of $c$. We specify it now.

        \begin{mathpar}
            \inferrule[2d]{c \whred c' \quad c' \whnorm c''}{c \whnorm c''} \and
            \inferrule[2e]{c \not \whred}{c \whnorm c} \and
            \inferrule[2f]{\quad}{(\lambda \alpha : k. \ c_1) c_2 \whred [c_2/\alpha]c_1} \and
            \inferrule[2g]{\quad}{\pi_1 \ \langle c_1, c_2 \rangle \whred c_1} \and
            \inferrule[2h]{\quad}{\pi_2 \ \langle c_1, c_2 \rangle \whred c_2} \and
            \inferrule[2i]{c_1 \whred c_1'}{c_1 \ c_2 \whred c_1' \ c_2}
        \end{mathpar} 

    Once weak-head normalization is performed, the syntactic range of constructors we need to consider is significantly reduced. Particularly, if paths are ``defined'' as follows:

        \[
            \begin{tabular}{lcl}
                $p$ & $::=$ & $\alpha \mid p \ c \mid \pi_1 \ c \mid \pi_2 \ c$
            \end{tabular}
        \]

    Then a weak-head normalized constructor has the form

        \[
            \begin{tabular}{lcl}
                $n$ & $::=$ & $p \mid c_1 \to c_2 \mid \forall \alpha : k. \ c$
            \end{tabular}  
        \]

    To get an intuition for what's going on here, the thing to understand is that paths have the form $\alpha \ c_1 \dots c_k$; that is, the outermost constructor is evaluated as far as possible. Comparing paths for equivalence is ``easy,'' particularly because if the outermost constructors differ no further work needs to be done. We can now define algorithmic path equivalence, $\Gamma \vdash n \patheq n' : k$. Notice that it's definition is mutually recursive with $\Gamma \vdash c \iff c' : k$. The rules are defined so that algorithmic path equivalence works as intended when given constructors in weak-head normal form.

        \begin{mathpar}
            \inferrule[2j]{\alpha : k \in \Gamma}{\Gamma \vdash \alpha \patheq \alpha : k} \and
            \inferrule[2k]{\Gamma \vdash p \patheq p' : k_1 \to k_2 \quad \Gamma \vdash c \iff c' : k_1}{\Gamma \vdash p \ c \patheq p' \ c' : k_2} \and
            \inferrule[2l]{\Gamma \vdash p \patheq p' : k_1 * k_2}{\Gamma \vdash \pi_1 \ p \patheq \pi_1 \ p' : k_1} \and
            \inferrule[2m]{\Gamma \vdash p \patheq p' : k_1 * k_2}{\Gamma \vdash \pi_2 \ p \patheq \pi_2 \ p' : k_2} \and
            \inferrule[2n]{\Gamma \vdash c_1 \iff c_1' : \type \quad \Gamma \vdash c_2 \iff c_2' : \type }{\Gamma \vdash (c_1 \to c_2) \patheq (c_1' \to c_2') : \type} \and
            \inferrule[2o]{\Gamma, \alpha : k \vdash c \iff c' : \type}{\Gamma \vdash (\forall \alpha : k. \ c) \patheq (\forall \alpha : k. \ c') : \type}
        \end{mathpar}

    Then, at last, we can state the following critical theorems:

        \begin{enumerate}
            \item The definition of $\Gamma \vdash c_1 \iff c_2 : k$ is inductive - in particular, proof search for a given $\Gamma \vdash c_2 \iff c_2 : k$ terminates. This is not true for $\Gamma \vdash c_1 \equiv c_2 : k$!
            \item (Soundness) If $\Gamma$ is well-formed, $\Gamma \vdash c_1 : k, \Gamma \vdash c_2 : k$, and $\Gamma \vdash c_1 \iff c_2 : k$, then $\Gamma \vdash c_1 \equiv c_2 : k$.
            \item (Completeness) If $\Gamma$ is well-formed, $\Gamma \vdash c_1 : k, \Gamma \vdash c_2 : k$ and $\Gamma \vdash c_1 \equiv c_2 : k$, then $\Gamma \vdash c_1 \iff c_2 : k$.
        \end{enumerate}

    At this point, you may be wondering, ``If the whole point of weak-head
    normalization is to get rid of function applications, why do we have an
    algorithmic path equivalence rule for \(p \ c \patheq p' \ c'\)?'' It would
    seem that this case of a path applied to a constructor would never come up.

    As it turns out, though, this is an important rule. When we weak-head
    reduce, there is no rule that applies to the case of variables. So we could
    have a weak-head normal form that looks like \(\alpha \ c\), for some
    \(\alpha\).

    An example of when such a case might occur is when attempting to derive
    \(\alpha : \type \to \type \vdash \alpha \iff \alpha\). A good exercise is
    to see which rules apply to derive this statement.

    \section{Type checking and kind checking}

    The typing and kinding rules are more straightforward than constructor equivalence. The judgements we will define are:

        \[ 
        \begin{tabular}{ll}
            Judgement & Description \\
            \hline
            $\Gamma^+ \vdash e^+ \synth k^-$ & Kind synthesis \\
            $\Gamma^+ \vdash e^+ \check k^+$ & Kind checking \\
            $\Gamma^+ \vdash e^+ \synth \tau^-$ & Type synthesis \\
            $\Gamma^+ \vdash e^+ \check \tau^+$ & Type checking
        \end{tabular}
        \]

    The rules for kinds are:
        \begin{mathpar}
            \inferrule[3a]{\alpha : k \in \Gamma}{\Gamma \vdash \alpha \synth k} \and
            \inferrule[3b]{\Gamma, \alpha : k \vdash c \synth k'}{\Gamma \vdash \lambda \alpha : k. \ c \synth k \to k'} \and
            \inferrule[3c]{\Gamma \vdash c_1 \synth k \to k' \quad \Gamma \vdash c_2 \check k}{\Gamma \vdash c_1 \to c_2 \synth k'} \and
            \inferrule[3d]{\Gamma \vdash c \synth k}{\Gamma \vdash c \check k} \and
            \inferrule[3e]{\Gamma \vdash c_1 \synth k_1 \quad \Gamma \vdash c_2 \synth k_2}{\Gamma \vdash \langle c_1, c_2 \rangle \synth k_1 * k_2} \and
            \inferrule[3f]{\Gamma \vdash c_1 \check \type \quad \Gamma \vdash c_2 \check \type}{\Gamma \vdash c_1 \to c_2 \synth \type} \and
            \inferrule[3g]{\Gamma, \alpha : k \vdash c \check \type}{\Gamma \vdash \forall \alpha : k. \ c \synth \type}
        \end{mathpar}
    For types:
        \begin{mathpar}
            \inferrule[3h]{x : \tau \in \Gamma}{\Gamma \vdash x \synth \tau} \and
            \inferrule[3i]{\Gamma \vdash \tau \check \type \quad \Gamma, x : \tau \vdash e \synth \tau'}{\Gamma \vdash \lambda x : \tau. \ e \synth \tau \to \tau'} \and
            \inferrule[3j]{\Gamma \vdash e_1 \synth \tau_1 \quad \tau \whnorm \tau \to \tau' \quad \Gamma \vdash e_2 \check \tau}{\Gamma \vdash e_1 \ e_2 \synth \tau'} \and
            \inferrule[3m]{\Gamma \vdash e \synth \tau \quad \tau \whnorm \forall \alpha : k. \ \tau' \quad \Gamma \vdash c \check k}{\Gamma \vdash e \ [c] \synth [c/\alpha]\tau'} \and
            \inferrule[3l]{\Gamma, \alpha : k \vdash e \synth \tau}{\Gamma \vdash \Lambda \alpha : t. \ e \synth \forall \alpha : t. \ \tau} \and
            \inferrule[3m]{\Gamma \vdash e \synth \tau' \quad \Gamma \vdash \tau \iff \tau' : \type}{\Gamma \vdash e \check \tau}

        \end{mathpar}
\end{section}
    
\end{document}
